Un compte github a été créé pour le pendule afin de pouvoir sauvegarder les fichiers en cas de problème. Les identifiants de ce compte et
des autres comptes liés à ce projet sont donnés dans le fichier texte "Comptes.txt" qui ne se trouve pas en annexe de ce document pour des
raisons de sécurité. Le fichier peut être trouvé dans la documentation rendue en annexe par fichier zip ou directement dans le dossier document
sur la carte RaspberryPi du pendule.

\section{Bouton marche/arrêt}
Un bouton lumineux permettant d'éteindre et d'allumer le système est présent sur le boîtier. Ce dernier est connecté sur la carte RaspberryPi comme
indiqué dans la figure \ref{fig:SchemaElec}. Deux fichiers sont utilisés pour gérer ce bouton: "listen-for-shutdown.sh" et "listen-for-shutdown.py".
Ces fichiers proviennent de l'exemple trouvé sur internet pour l'ajout d'un bouton sur la carte RaspberryPi \cite{ButtonAdd}. Le fichier qui se
termine par ".sh" est un fichier \textit{shell script}, un fichier exécutable sous Linux. Ce fichier s'occupe de lancer et arrêter le
fichier qui se termine par ".py". Ce fichier est codé en python et met le pin GPIO3 en entrée et attend un flanc descendant pour arrêter la
carte RaspberryPi. Le pin GPIO3 est spécial car c'est celui qui est capable de réveiller la carte lorsqu'il reçoit un flanc descendant.\\

La méthode utilisée dans le tutoriel pour que le fichier "listen-for-shutdown.sh" se lance à l'allumage du système ne fonctionne pas. Un autre
tutoriel visant à lancer un programme au démarrage \cite{ScriptStartup} à permis au code de se lancer correctement après allumage. Cette méthode
utilise un fichier crontab qui va exécuter le fichier se finissant par ".sh" à intervalle régulier.\\

Le bouton possède aussi une LED qui permet de savoir l'état du système. Un tutoriel sur internet \cite{LEDAdd} indique comment ajouter une LED
sur la RaspberryPi. Plusieurs méthodes sont mentionnées mais celle utilisée consiste à brancher l'anode de la LED sur le pin de sortie de la
communication UART. La trame UART étant définie comme à l'état logique 1 tant qu'il n'y a pas de message la LED reste allumée. Cette méthode
est simpliste mais la communication UART n'est pas utilisée lors de ce projet et ceci permet d'avoir un indicateur correct pour pouvoir couper
l'alimentation de la carte RaspberryPi étant donné que la LED ne s'éteind que lorsque la carte est complètement à l'arrêt. La cathode, elle,
est connectée sur un pin GND.

\section{Initialisation EPOS4}
Le contrôleur EPOS4 peut se brancher sur un ordinateur personnel avec un connecteur micro USB afin d'effectuer des tests. Pour cela, un logiciel
appelé EPOS Studio qui est téléchargeable sur le site Maxon\cite{Maxon} est utilisé. Il permet de modifier les registres dans les contrôleurs EPOS
afin de les régler pour le type de moteur qui sera contrôlé comme illustré ci-dessous.

\begin{figure}[H]
    \centering
    \includegraphics[width = 0.8\textwidth]{assets/figures/Registres.png}
    \caption{Registres de l'EPOS4}
    \label{fig:Registres}
\end{figure}

\subsection{Démarrage}
Pour aider avec la gestion des registres des \textit{wizards} sont mis à disposition. Le premier s'appelle \textit{Startup wizard}, il permet
de configurer les paramètres de notre moteur et des capteurs du système. Une image du \textit{wizard} est donnée ci-après. La configuration
complète est présente dans l'annexe "ConfigurationWizards.docx" dans le dossier du projet.

\begin{figure}[H]
    \centering
    \includegraphics[width = 0.8\textwidth]{assets/figures/StartupWizard.png}
    \caption{Fenêtre de réglage des paramètres du moteur et des capteurs}
    \label{fig:StartWizard}
\end{figure}

Cependant, un problème est présent lors de la configuration des capteurs. La description du problème et la démarche qui devra être suivie pendant
le mois d'août pour le résoudre sont spécifiés au chapitre \ref{sec:ProbCapt}. Ce problème empêche de pouvoir contrôler le moteur et donc de
pouvoir communiquer en CANopen avec.\\

\subsection{Régulation}
Le second \textit{wizard} s'appelle \textit{"Regulation Tuning"} et permet d'ajuster les paramètres de régulation du système. L'illustration
suivante montre à quoi ressemble la fenêtre.

\begin{figure}[H]
    \centering
    \includegraphics[width = 0.8\textwidth]{assets/figures/RegulationTuning.png}
    \caption{Fenêtre de réglage pour la régulation d'un moteur}
    \label{fig:RegTune}
\end{figure}

Etant donné que les capteurs ne peuvent pas être utilisés dans cette configuration, le seul test utilisable sur cet onglet pour l'instant
est la régulation de courant. Il suffit de cliquer sur \textit{"Auto-tune"} pour que le système trouve les valeurs du régulateur de courant.\\

\subsection{CANopen}

\subsection{Export de paramètres}

\section{Communication CAN}

\section{Régulation}

