% Résumé du projet
Ce projet porte sur la fabrication et la programmation d'un pendule inverse. Celui-ci doit être assemblé avec les demandes
suivantes:
\begin{itemize}
    \item La structure principale est faite de profilés en aluminium.
    \item Un moteur linéaire sans coeur en fer est utilisé pour entrainer le système avec le moins de frottements possible.
    \item La tige du pendule est illuminée par des leds.
\end{itemize}

La programmation, elle, sera faite à l'aide de boucles de régulations \gls{PID}. Ces dernières permettront de maintenir la tige du pendule
en équilibre instable ainsi qu'au centre de la structure.\\

Le travail à fournir est donc le suivant:
\begin{itemize}
    \item Développement de la structure en profilé.
    \item Création de l'entraînement mécanique du système avec guidage.
    \item Mise en place de la mesure de position linéaire du pendule ainsi que de la position angulaire de la tige.
    \item Création d'un boîtier d'alimentation électrique permettant au système de fonctionner avec une prise 230~V classique.
    \item Commande et assemblage des différentes pièces nécessaire du système.
    \item Programmation des boucles de régulations du pendule.
\end{itemize}