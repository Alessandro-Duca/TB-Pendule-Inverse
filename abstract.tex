% Résumé du projet
Ce projet porte sur la fabrication et la programmation d'un pendule inverse. Celui-ci doit être assemblé en suivant des directives précises.
Tout d'abord, la structure principale doit être faite de profilés en aluminium. Ensuite, un moteur linéaire sans coeur en fer est utilisé pour
entrainer le système avec le moins de frottements possible. Un contrôleur moteur et une carte RaspberryPi qui le dirige par communication \gls{CAN}
doivent être utilisés pour faire fonctionner le moteur. Enfin, la tige du pendule doit être illuminée par des leds.\\

La programmation, quant à elle, sera faite à l'aide de boucles de régulations \gls{PID}. Ces dernières permettront de maintenir la tige du pendule
en équilibre instable ainsi qu'au centre de la structure.\\

Le travail à fournir pendant la durée de ce travail de Bachelor comporte plusieurs étapes clés. Tout d'abord, le développement de la structure
en profilé et la création de l'entraînement mécanique du système avec guidage. Ensuite, la mise en place de la mesure de position linéaire
du pendule ainsi que de la position angulaire de la tige. La création d'un boîtier d'alimentation électrique permettant au système de
fonctionner avec une prise 230~V classique est aussi nécessaire. Une fois l'assemblage complet bien défini, les étapes suivantes sont la commande
et l'assemblage des différentes pièces nécessaire du système. Enfin, la dernière étape consiste à programmer les boucles de régulations du
pendule.\\

Malgré l'apparition de quelques problèmes qui seront réglés pendant le mois d'août, le projet est déjà bien avancé. Le développement d'une
structure rigide et esthétique, l'assemblage de toutes les pièces commandées et développées, le câblage interne et externe au boîtier et la mise
en tension du système ont été effectués avec succès lors du projet.