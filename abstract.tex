% Résumé du projet
Ce projet porte sur la fabrication et la programmation d'un pendule inverse. Celui-ci doit être assemblé en suivant des directives précises.
Tout d'abord, la structure principale doit être faite de profilés en aluminium. Ensuite, un moteur linéaire sans coeur en fer est utilisé pour
entrainer le système avec le moins de frottements possible. Enfin, la tige du pendule doit être illuminée par des leds.\\

La programmation, elle, sera faite à l'aide de boucles de régulations \gls{PID}. Ces dernières permettront de maintenir la tige du pendule
en équilibre instable ainsi qu'au centre de la structure.\\

Le travail à fournir pendant la durée de ce travail de Bachelor comporte plusieurs étapes clés. Tout d'abord, le développement de la structure
en profilé et la création de l'entraînement mécanique du système avec guidage. Ensuite, il faut mettre en place la mesure de position linéaire
du pendule ainsi que de la position angulaire de la tige. Il faut aussi créer un boîtier d'alimentation électrique permettant au système de
fonctionner avec une prise 230~V classique. Une fois l'assemblage complet bien défini, il faut passer à la commande et l'assemblage des
différentes pièces nécessaire du système. Pour finir, il reste la programmation des boucles de régulations du pendule.