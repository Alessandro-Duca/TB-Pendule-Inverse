%Introduction
Un pendule inversé est un système instable qui vise à maintenir une tige dans sa position d'équilibre instable, c'est-à-dire à la verticale
maintenue par le bas. Sans un système pour maintenir cette position, la moindre déviation par rapport à la position d'équilibre instable va
précipiter le système vers sa position d'équilibre stable, c'est à dire à la verticale maintenue par le haut. La régulation du système va
permettre de maintenir la position d'équilibre instable à l'aide du contrôle d'un élément du système. Il existe différents éléments qui sont
contrôlables pour maintenir la position d'équilibre instable du pendule, mais dans le cas de ce projet, un déplacement horizontal de la base
sera utilisé. L'objectif principal est d'avoir une maquette qui illustre la régulation afin de pouvoir l'expliquer facilement. Le projet est
très complet car plusieurs domaines vont être nécessaires pour en venir à bout. Tout d'abord, le développement mécanique avec la structure
qui permet le maintien ainsi que le déplacement horizontal du système. Ensuite, la programmation permettra de déplacer le système afin qu'il
reste en équilibre et ce malgré des perturbations, grâce à la régulation \gls{PID}. Enfin, la mise en place
d'un boîtier d'alimentation électrique fonctionnant avec une prise électrique classique T13 de 230V est aussi présente.\\

Le projet abouti serait une structure du pendule entièrement assemblée qui peut connaître sa position linéaire et angulaire, être commandée
en déplacement horizontal et peut maintenir la position d'équilibre instable de la tige au centre du système.

\section{Contexte}
L'\acrlong{iAi} a proposé de fabriquer un pendule inversé dans le cadre d'un travail
de Bachelor. Le pendule pourra être utilisé par la suite comme exemple de régulation lors d'événements servant à présenter l'\acrshort{Heig}.
Dans ce but-là, une place en C26 a été mise à disposition pour travailler et un budget de maximum 10'000 CHF a été alloué. Un échéancier
avec les dates pour les rendus est disponible dans l'annexe E. Ce document est à rendre pour le 28 juillet 2023. Toute la période d'août entre
le rendu de ce document et la défense du travail est disponible pour avancer le projet si nécessaire.