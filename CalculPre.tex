Les calculs suivant sont nécessaire à la prise de décision pour les solutions.

\section{Valeurs utilisées}\label{sec:ValUtil}
On retrouve dans le tableau suivant les valeurs constantes que l'on connaît au préalable et qui seront utilisées dans les calculs.
On approxime une valeur d'accélération maximale voulue de 30~$m/s^2$ et une vitesse maximale de 5~m/s pour une course d'environ 1~m.
Ces valeurs pourront être ajustées si jamais les calculs montrent qu'elles ne sont pas atteignables.

\begin{table}[H]
    \centering
    \caption{Valeurs utilisées pour les calculs}
    \label{tab:ValUtil}
    \begin{tabular}{|l|l|l|l|}
        \hline
        \textbf{Donnée}             & \textbf{Lettre} & \textbf{Valeur} & \textbf{Unité} \\ \hline
        Accélération max du chariot & a               & 30              & $m/s^2$        \\ \hline
        Vitesse max du chariot      & v               & 5               & $m/s$          \\ \hline
        Course du chariot           & l               & 1               & $m$            \\ \hline
    \end{tabular}%
\end{table}

On estime la masse totale qui bouge à l'aide des approximations suivantes en prennant les éléments mentionnés dans le chapitre \ref{sec:PartMob}
et représentés dans la figure \ref{fig:StructPrelim}:
\begin{itemize}
    \item Tige en polycarbonate de 10~mm de diamètre: $\sim$50~g
    \item Glider moteur linéaire: $\sim$200~g
    \item Chariot du guidage linéaire: $\sim$100~g
    \item Resolver: $\sim$100~g
    \item Tête de lecture encodeur linéaire: $\sim$70~g
    \item Pièces créées en aluminium: $\sim$250-350~g
\end{itemize}

On aura donc une masse totale m en mouvement pouvant aller de 770 à 870~g. La valeur maximale sera utilisée pour déterminer la force maximale F
que le moteur doit appliquer pour atteindre l'accélération voulue en utilisant la formule suivante:

\begin{equation}\label{eq:ForceMot}
    F = m \cdot a = 0.87 \cdot 30 = 26.1~N
\end{equation}

Le moteur choisi devra appliquer une force d'au moins 26.1~N pour pouvoir obtenir l'accélération recherchée sur le pendule.\\

On peut aussi vérifier que la course du moteur est suffisante pour atteindre la vitesse maximale du système. On ne peut accélérer que sur la
moitié de la course, car il faut décélérer sur la deuxième moitié afin d'éviter les impacts. On obtient donc les formules suivantes:

\begin{align}\label{eq:TempsMouv}
    \begin{split}
        x(t) = \frac{l}{2} = \frac{1}{2} \cdot a \cdot t^2 + v_0 \cdot t + x_0 \; avec \; v_0 = 0 \; et \; x_0 = 0 \\ \Rightarrow t = \sqrt{\frac{2 \cdot l}{2 \cdot a}} = \sqrt{\frac{2 \cdot 1}{2 \cdot 30}} = 183~ms
    \end{split}
\end{align}

\begin{equation}
    v(t) = a \cdot t + v_0 = 30 \cdot 0.183 + 0 = 5.48~m/s
\end{equation}

La valeur de vitesse maximale est bien atteignable avec cette course pour le chariot.